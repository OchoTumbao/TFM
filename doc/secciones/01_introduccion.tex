\chapter{Introducción}

\section{Contexto y Motivación}

\subsection{Inteligencia Artificial y Videojuegos}

El desarrollo de agentes Inteligentes para jugar a juegos es un campo que ha estado ligado a la Inteligencia Artificial desde sus inicios. Existen papers datados de 1953 donde se habla del uso de ordenadores para jugar a juegos \cite{Turing_1953a}. Desde entonces, hemos visto un gran avance en este campo y contamos con agentes como Alphazero \cite{1712.01815} o AlphaGo \cite{AlphaGo} capaces de jugar a juegos como el ajedrez o el Go a un nivel igual o superior que el de los jugadores humanos profesionales. 

Paralelamente, el campo de los videojuegos también ha evolucionado de forma notable, siendo a dia de hoy una de las industrias que más dinero mueve en el mundo. Sin embargo, más alla de eso, los videojuegos son una forma de arte y entretenimiento digital que combina narrativa, diseño visual, música y jugabilidad junto a la tecnología. Beneficiandose historicamente de los avances en computación. 

Existe una sinergia a destacar entre los videojuegos y la computación que es el uso que se le ha dado dentro del medio a las herramientas de Inteligencia Artificial. Donde los primeros se han beneficiado de todos los avances de la Inteligencia Artificial para mejorar la experiencia de los jugadores, entregando enemigos más inteligentes o aliados más capaces.Mientras que por otra parte, la Inteligencia Artificial ha encontrado en los videojuegos un campo de pruebas ideal para probar sus algoritmos y técnicas.

Partiendo desde los primeros agentes inteligentes que jugaban a los juegos de mesa clasicos, Arthur Samuel desarrollo en 1959 un agente capaz de jugar al juego de las damas inventando en el proceso el concepto de \textit{aprendizaje por refuerzo} \cite{CheckersSamuel1959}. Mientras paralelamente el ajedrez se convertía en un modelo al que muchos de los avances de la Inteligencia Artificial se enfrentaban para comprobar su eficacia. Fue en 1997 cuando Deep Blue gano por primera vez a un campeón mundial de ajedrez Garry Kasparov. Y el campo no ha dejado de avanzar desde entonces, en la actualidad puedes jugar libremente contra agentes Inteligentes que superan a los mejores jugadores humanos en juegos como el ajedrez o el Go que hemos mencionado anteriormente.

Por otro lado, los videojuegos representan otro tipo de reto para los agentes Inteligentes. Presentan entornos donde la acción no tiene porque ocurrir en turnos, donde la información puede ser parcial. Un avance muy importante en este campo fue cuando en 2014 un agente Inteligente desarrollado por Google DeepMind fue capaz de jugar y superar a los mejores jugadores humanos en diferentes videojuegos de la consola Atari 2600 \cite{Atari2600}. Pero la cosa no acaba ahí, sigue existiendo un interes de la comunidad cientifica por desarrollar agentes Inteligentes capaces de jugar a videojuegos más complejos. Avances como el de OpenAI con su agente capaz de jugar a Dota 2 en 2019 \cite{OpenAIDota2} o el de DeepMind con su agente capaz de jugar a StarCraft II en  \cite{StarCraftII} son ejemplos de ello.

No solo existe interes por desarrollar agentes Inteligentes para jugar a videojuegos, sino que también se han usado tecnicas de Inteligencia Artificial para generar contenido proceduralmente dentro de los videojuegos. Juegos como \textit{No Man's Sky} o \textit{Minecraft} son ejemplos de videojuegos que usan Inteligencia Artificial para generar su mundo y su mapa proceduralmente. Esto permite a los jugadores explorar mundos virtuales casi infinitos, donde cada partida es única y diferente. E incluso en otros videojuegos como \textit{Middle Earth: Shadow of Mordor} se ha usado Inteligencia Artificial para crear un sistema de enemigos que aprenden y evolucionan a partir de las acciones del jugador, creando una experiencia de juego más dinámica y personalizada \cite{NemesisSystem}. 

En la actualidad con el auge de los modelos generativos existe un claro interes por parte de las empresas de videojuegos por integrar estos modelos en sus productos. Sea para la generación de contenido, para otorgar nuevas experiencias a los jugadores o para abaratar costes. Existe cierto debate en la comunidad sobre la legitimidad que aún se debate entre las posibilidades que esto ofrece y las cuestiones éticas que plantean. Sin embargo como hemos podido ver, la Inteligencia Artificial ha estado presente en los videojuegos desde sus inicios y ha evolucionado junto a ellos, siendo dos campos que se retroalimentan mutuamente.

\subsection{Pokémon y su comunidad competitiva}

\textit{Pokémon} es una franquicia de videojuegos con un impacto significativo en la cultura popular \cite{Bokksu}. Desde su origen en 1996 con el lanzamiento en Japon de \textit{Pokémon Red} y \textit{Pokémon Green}, la franquicia creció hasta llegar a ser un fenomeno global que perdura hasta nuestros días. Salieno del propio medio de los videojuegos, estableciendo su presencia mediante productos como series de televisión, preliculas, juegos de cartas y merchandising. 

La estructura de los videojuegos de Pokémon gira en torno a las propias criaturas, los Pokémon. Perteneciendo al genero de los juegos de rol y al subgenero de los "Monster Taming Games" o juegos de captura de criaturas Inspirado por juegos como la saga Shin Megami Tensei. En estos juegos los jugadores asumen el papel de entrenadores Pokémon, que viajan por un mundo ficticio habitado por estas criaturas. Siendo su objetivo formar un equipo de Criaturas, entrenarlos y competir contra otros entrenadores que también forman sus propios equipos.

La jugabilidad entonces radica precisamente en el proceso de creación de un equipo y de las propias batallas entre Pokémon. Estas se desarrollan por turnos, donde cada jugador elige una acción para su Pokémon y segun una serie de reglas que explicaremos en futuros apartados se resuelve el resultado de la acción. La complejidad de este sistema de combate radica en la gran cantidad de factores que influyen en la resolución de las batallas. Siendo cada especie de Pokémon única, e incluso habiendo diferencias entre individuos de la misma especie. Cada Pokémon tiene sus propios atributos, movimientos, habilidades y tipos, lo que añade una capa de estrategia a las batallas. El juego es extremadamente complejo.

Para tener una referencia sobre la complejidad del juego podemos aprovecharnos de que dado su caracter discreto, donde las acciones se desarrollan por turnos. Podemos calcular de forma similar a lo que se hace con otros juegos de mesa que si son más conocidos como las damas, el ajedrez o el Go la complejidad espacial del juego. Lo que quiere decir el numero de posibles estados que puede alcanzar una batalla promedio y compararlo con estos otros juegos. Este calculo ya ha sido realizado como parte del desarrollo del proyecto FutureSight del que hablaremos más adelante arrojando un resultado de $10^{358}$ posibles estados \cite{FutureSightTotalBattleSituations}. Para poner esto en perspectiva, el Go tiene una complejidad espacial de $10^{172}$ y el Ajedrez de $10^{123}$ \cite{GoComplexity}.

Esta complejidad y la popularidad de la saga ha llevado a la creación de una comunidad competitiva en torno a ella. La cual organiza torneos y competiciones donde los jugadores se enfrentan entre si en batallas Pokémon. En la actualidad estas competiciones son eventos masivos que atraen a miles de jugadores y espectadores. El ultimo evento de este tipo, el campeonato mundial de Pokémon de 2024 se celebro en Honolulu, Hawaii conto con un pico de espectadores de 123.900 personas \cite{2024PokemonWorldChampionshipsViewership}.

En la actualidad existen dos asociaciones que gestionan el juego competitivo de Pokémon. Por un lado existe el formato VGC (Video Game Championships) que es el formato oficial de la compañía The Pokémon Company. En este formato es la propia compañia la que decide que reglas y restricciones se aplican y se juega en las competiciones oficiales con los propios juegos de la franquicia. Mientras que por otro lado existe Smogon que es una agrupación de jugadores que desarrollan sus propias reglas y formatos de juego. Este formato se suele jugar en simuladores web como Pokémon Showdown

\subsection{Pokémon y la Inteligencia Artificial}

\textit{Pokémon} como videojuego que es ha utilizado Inteligencia Artificial desde sus inicios. Los combates que juegas en el juego cuando no estas jugando contra otros jugadores son gestionados por un algoritmo que determina las acciones que toma el oponente. Este algoritmo se revisa en cada entrega de la saga, pero en su esencia más basica y en su forma original es simplemente un algoritmo de decisión basado en reglas que es más o menos complejo en función de la dificultad pretendida para ese combate \cite{PokemonRedAIAlgorithm}.

Además del propio uso dentro de la saga, la comunidad competitiva ha desarrollado también un interes por los agentes Inteligentes para jugar.Por ejemplo en 2017 Smogon, la agrupación fan de jugadores competitivos, celebro la Showdown AI Competition \cite{ShowdownAICompetition}. Un evento donde se invitaba a los participantes a desarrollar agentes Inteligentes capaces de jugar en el simulador Pokémon Showdown. 

Se han abordado diferentes enfoques para desarrollar agentes Inteligentes enfocados en batallas Pokémon. En 2014 se presente Percymon \cite{Percymon}, un agente Inteligente que usaba un algoritmo Minimax clasico con profundiad 2 para jugar en el simulador Pokémon Showdown. y el campo no se ha quedado estatico desde entonces. En la actualidad se estan probando tanto acercamientos basados en aprendizaje por refuerzo \cite{VGCDeepReinforcementLearning} como en algoritmos de busqueda potenciados por el uso de LLM \cite{Pokechamp}

También se han desarrollado entornos donde poder probar estos agentes. Siendo de particular interes para nuestro trabajo el entorno  VGC AI Framework \cite{VGCAICompetition} este entorno ofrece una ligera simplificación de las reglas del juego, permitiendo a los agentes jugar en un entorno más controlado y simplificado. Este entorno es una herramienta valiosa para el desarrollo y prueba de agentes Inteligentes en el contexto de batallas Pokémon. Además permite desarrollar facilmente competiciones entre agentes y ofrece una API que facilita el desarrollo de los propios agentes. Este entorno lleva siendo usado anualmente para una competición de agentes inteligentes para Pokémon en la Conference on games de la IEE, ofreciendo no solo un entorno centrado en las batallas si no también de construcción de equipos.

\subsection{La competición VGC AI Competition}

Antes hemos mencionado el entorno VGC AI Framework, y la competición VGC AI Competition. 
Para ponernos en contexto, La VGC AI Competition es una competición que se celebra anualmente en la Conference on Games de la IEE, organizada por Simao Reis. Esta competición orientada a los investigadores y desarrolladores de Inteligencia Artificial en videojuegos plantea un marco de reglas sobre las cuales se evaluan los agentes. Para esto crea 3 tipos de competiciones o "tracks" como las llaman ellos.
\begin{itemize}
    \item \textbf{Battle Track:} En esta competición los participantes deben desarrollar un agente capaz de jugar batallas Pokémon. Los equipos son generados aleatoriamente en cada combate y por tanto no hay que preocuparse sobre la construcción de equipos.
    \item \textbf{Championship Track:} En esta competición se genera un roster de criaturas y los participantes deben desarrollar un agente capaz de construir el mejor equipo posible con esas criaturas y luego jugar batallas con él.
    \item \textbf{Meta-Game Balance Track:} En este caso el objetivo es desarrollar un agente capaz de generar un meta juego equilibrado. Es decir debe crear un roster de criaturas lo suficientemente diverso y equilibrado para cumplir unos criterios de equilibrio.
\end{itemize}

En nuestro proyecto nos vamos a centrar en la primera de ellas. Por tanto nuestro agente solo tendrá que preocuparse por ser capaz de tomar las decisiones correctas en cada turno de la batalla y de una fase que ellos llaman "de seleccion" donde se muestra el equipo del oponente y cada jugador elige un subconjunto de criaturas para usar en la batalla.

Para gestionar todas estas batallas, se utiliza el entorno VGC AI Framework, el cual explicaremos detalladamente en otra sección.

\section{Propuesta}
Conociendo el contexto. Nuestro objetivo es utilizar el entorno VGC AI Framework para realizar un estudio comparativo entre diferenetes enfoques y tecnicas de Inteligencia Artificial para desarrollar un agente capaz de participar en el battle track de la VGC AI Competition. Las diferentes tecnicas que vamos a estudiar son las siguientes:

\begin{itemize}
    \item \textbf{Estrategias de Coevolucion:} Basandonos en trabajo previo de el tutor Pablo García Sánchez, vamos a implementar un agente que utilice una función de fitness para calcular la calidad del estado del juego despues de cada acción. Esta función de fitness se optimizara por medio de una estrategia de coevolución, donde el agente se enfrentara a otros agentes para mejorar su rendimiento.
\end{itemize}